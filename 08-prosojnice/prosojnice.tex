\documentclass{beamer}
% Naloga 1.3.1: Za dokument uporabite razred `beamer'.
% Ne dodajajte nastavitve za velikost pisave, kot je bila v datoteki `5-prosojnice.tex`.

% Naloga 1.3.2: vključite paket `predavanja'.
\usepackage{predavanja}
% Naloga 1.3.3: definirajte okolji `definicija' in `izrek'.
% Namig: z iskanjem po datotekah (Ctrl+Shift+F oz. Cmd+Shift+F) 
% poiščite niz `{definicija}' ali niz `{izrek}'.
\usepackage{amsthm}
\usepackage{amsfonts}
\usepackage{amsthm}

\newtheorem{definicija}{Definicija}
\newtheorem{izrek}{Izrek}

\usepackage{tikz}
\usetikzlibrary{math}

\usepackage{pgfplots}
\usepgfplotslibrary{external}

\usepackage{array} %za tebele da se stolpci postopoma odkriva

\title{Matematični izrazi in uporaba paketa \texttt{beamer}}
\subtitle{\emph{Matematičnih} nalog ni treba reševati!}
\author{Fakulteta za matematiko in fiziko}
\date{}

\begin{document}

% Naloga 1.3.4: pripravite naslovno stran z vsebino:
\frame{\titlepage}

\begin{frame}{Kratek pregled} % [pausesections]
    \tableofcontents[hideallsubsections]
\end{frame}

\section{Paket \texttt{beamer}}


%\begin{frame}{Paket \texttt{beamer}}
%   Osnovne informacije o beamerju
%\end{frame}

\section{Paket \texttt{beamer}}
\begin{frame}{Posebnosti prosojnic}
%    Za lepše matematične izraze
%Za prosojnice je značilna uporaba okolja \texttt{frame},
%s katerim definiramo posamezno prosojnico,
\pause
%postopno odkrivanje prosojnic,
\pause

%ter nekateri drugi ukazi, ki jih najdemo v paketu \texttt{beamer}.
\pause

\begin{exampleblock}{Primer}
    %Verjetno ste že opazili, da za naslovno prosojnico niste uporabili
    %ukaza \texttt{maketitle}, ampak ukaz \texttt{titlepage}.
\end{exampleblock}
\end{frame}

\begin{frame}{Poudarjeni bloki}
   
    \begin{block}{Opomba}
        \end{block}

        Okolja za poudarjene bloke so \texttt{block}, \texttt{exampleblock} in \texttt{alertblock}.
\begin{alertblock}{Pozor!}
%Začetek poudarjenega bloka (ukaz begin) vedno sprejme dva parametra: okolje in naslov bloka. Drugi parameter (za naslov) je lahko prazen.
\end{alertblock}

\end{frame}

\begin{frame}{Tudi v predstavitvah lahko pišemo izreke in dokaze}
\begin{izrek}
    Praštevil je neskončno mnogo.
\end{izrek}

\begin{proof}
    Denimo, da je praštevil končno mnogo.
   
    \begin{itemize}[<+->]
        \item Naj bo $p$ \alert<4>{največje} praštevilo.
        \item Naj bo $q$ produkt števil $1$, $2$, $\dots$, $p$.
        \item Število $q+1$ ni deljivo z nobenim praštevilom, torej je $q+1$ praštevilo.
        \item To je protislovje, saj je $q+1 > p$.
    \end{itemize}
\end{proof}
\end{frame}


% - naslov: Matematični izrazi in uporaba paketa \texttt{beamer}
% - podnaslov: \emph{Matematičnih} nalog ni treba reševati!
% - inštitut: Fakulteta za matematiko in fiziko
% - datum: naj se ne izpiše; to dosežete z ukazom \date{}.
% Zgornje podatke nastavite z ukazi kot v dokumentih razreda `article`.
% Več o tem, kako se naredi naslovno stran, si preberite na naslovu na naslovu:

% To stran preberite do vključno razdelka "Creating a table of contents".
% Ukaz `\titlepage` deluje podobno kot ukaz `\maketitle`, ki ste ga že srečali.

% Naloga 1.3.5: pripravite kazalo vsebine.
% 1. Naslov prosojnice, s kazalom vsebine naj bo "Kratek pregled"
% 2. S pomožnim parametrom `pausesections' (v oglatih oklepajih) 
%    določite, da naj se kazalo vsebine odkriva postopoma.
%    Poglejte, kako deluje ta ukaz.
% 3. Ker ni videti preveč lepo, pomožni parameter zakomentirajte.

\section{Paket \texttt{beamer}}
\input{prosojnice/1-paket-beamer.tex}
\section{Paketa \texttt{amsmath} in \texttt{amsfonts}}
\input{prosojnice/2-paketa-amsmath-amsfonts.tex}

\begin{frame}{Matrike}
    \frametitle{Matrike}
    Izračunajte determinanto
    \[
    \begin{vmatrix}
    -1 & 4 & 4 & -2 \\
    1 & 4 & 5 & -1 \\
    1 & 4 & -2 & 2 \\
    3 & 8 & 4 & 3
    \end{vmatrix}\]

    V pomoč naj vam bo Overleaf dokumentacija o matrikah:

\href{https://www.overleaf.com/learn/latex/Matrices}{\beamergotobutton{Matrices}}
\end{frame}

% PRVA PROSOJNICA - samo prva enačba
\begin{frame}{Okolje align in align*}
Dokaži \emph{binomsko formulo}: za vsaki realni števili $a$ in $b$ in za vsako naravno število $n$ velja

\[
(a + b)^n = \cdots
\]
\[
= \sum_{k=0}^n \binom{n}{k} a^{n-k} b^k
\]
\end{frame}

% DRUGA PROSOJNICA - prva + druga enačba
\begin{frame}{Okolje align in align*}
Dokaži \emph{binomsko formulo}: za vsaki realni števili $a$ in $b$ in za vsako naravno število $n$ velja

\[
(a + b)^n = (a + b)(a + b) \ldots (a + b)
\]
\[
= \sum_{k=0}^n \binom{n}{k} a^{n-k} b^k
\]
\end{frame}

% TRETJA PROSOJNICA - vse tri enačbe
\begin{frame}{Okolje align in align*}
Dokaži \emph{binomsko formulo}: za vsaki realni števili $a$ in $b$ in za vsako naravno število $n$ velja

\[
(a + b)^n = (a + b)(a + b) \ldots (a + b)
\]
\[
= a^n + na^{n-1}b + \ldots + \binom{n}{k}a^{n-k}b^k + \ldots + nab^{n-1} + b^n
\]
\[
= \sum_{k=0}^n \binom{n}{k}a^{n-k}b^k
\]
\end{frame}

\begin{frame}{Še ena uporaba okolja \texttt{align*}}
\textbf{Nariši grafe funkcij:}

\begin{align*}
y &= x^2 - 3|x| + 2 &\qquad y &= 3\sin(\pi + x) - 2 \\
y &= \log_2(x - 2) + 3 &\qquad y &= 2\sqrt{x^2 + 15} + 6 \\
y &= 2^{x-3} + 1 &\qquad y &= \cos(x - 3) + \sin^2(x + 1)
\end{align*}

\end{frame}

\begin{frame}{Okolje multline}
Poišči vse rešitve enačbe

\begin{multline*}
(1 + x + x^2) \cdot (1 + x + x^2 + x^3 + \ldots + x^9 + x^{10}) = \\
= (1 + x + x^2 + x^3 + x^4 + x^5 + x^6)^2.
\end{multline*}

\end{frame}

\begin{frame}{Okolje cases}
\frametitle{Dana je funkcija}

\[
f(x, y) =
\begin{cases}
\dfrac{3x^2y - y^3}{x^2 + y^2}, & (x, y) \neq (0, 0), \\
a, & (x, y) = (0, 0).
\end{cases}
\]

\begin{itemize}
\item \textbf{Določi} $a$, tako da izračunaš limito $\displaystyle \lim_{(x, y) \to (0, 0)} f(x, y)$.
\item \textbf{Izračunaj parcialna odvoda} $f_x(x, y)$ in $f_y(x, y)$.
\end{itemize}

\end{frame}


\section[Matematika, 1. del\\\large{Analiza, logika, množice}]{Matematika, 1. del}
\input{prosojnice/3-analiza-logika-mnozice.tex}

\begin{frame}{Logika in množice}
\begin{enumerate}
\item
Poišči preneksno obliko formule  
\[ \exists x : P(x) \land \forall x : Q(x) \Rightarrow \forall x : R(x). \]

\item
Definiramo množici $A = [2, 5]$ in $B = \{0, 1, 2, 3, 4 \ldots\}$.  
V ravnino nariši:
\begin{enumerate}
   \item $A \cap B \times \emptyset$
   \item $(A \cup B)^c \times \mathbb{R}$
\end{enumerate}

\item
Dokaži:
\begin{itemize}
    \item $(A \Rightarrow B) \sim (\neg B \Rightarrow \neg A)$
    \item $\neg(A \lor B) \sim \neg A \land \neg B$
\end{itemize}
\end{enumerate}
\end{frame}

\begin{frame}{Analiza}
\begin{enumerate}
\item
Pokaži, da je funkcija $x \mapsto \sqrt{x}$ enakomerno zvezna na $[0, \infty)$.

\item
Katero krivuljo določa sledeč parametričen zapis?
\[
   x(t) = a \cos t, \quad
   y(t) = b \sin t, \quad
   t \in [0, 2\pi]
\]

\item
Pokaži, da ima $f(x) = 3x + \sin(2x)$ inverzno funkcijo in izračunaj $(f^{-1})'(3\pi)$.

\item
Izračunaj integral
\[
\int \frac{2 + \sqrt{x+1}}{(x+1)^2 - \sqrt{x+1}} \, dx
\]

\item
Naj bo $g$ zvezna funkcija. Ali posplošeni integral
\[
\int_0^1 \frac{g(x)}{x^2} \, dx
\]
konvergira ali divergira? Utemelji.
\end{enumerate}
\end{frame}

\begin{frame}{Kompleksna števila}
\begin{enumerate}
\item
Naj bo $z$ kompleksno število, $z \neq 1$ in $|z| = 1$.  
Dokaži, da je število \(\displaystyle i \frac{z+1}{z-1} \) realno.

\item
Poenostavi izraz:
\[
\frac{3+i}{2-2i} + \frac{7i}{1+i} - \left(1 + \frac{i-1}{4} - \frac{5}{2-3i}\right)
\]
\end{enumerate}
\end{frame}


\section{Stolpci in slike}
\input{prosojnice/4-stolpci-slike.tex}

\section{Paket \texttt{beamer} in tabele}
\input{prosojnice/5-beamer-tabele.tex}

\section[Matematika, 2. del\\\large{Zaporedja, algebra, grupe}]{Matematika, 2. del}
\input{prosojnice/6-zaporedja-algebra-grupe.tex}

\end{document}